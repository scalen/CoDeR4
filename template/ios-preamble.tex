\documentclass{iosart2c}


\usepackage[T1]{fontenc}
\usepackage{times}%

\usepackage{natbib}
%\usepackage[dvips]{hyperref}
\usepackage{amsmath}
\usepackage{dcolumn}
%\usepackage{endnotes}
%\usepackage{graphics}

\newcolumntype{d}[1]{D{.}{.}{#1}}

\firstpage{1} \lastpage{5} \volume{1} \pubyear{2009}

\usepackage{cuted}

\usepackage{amssymb}
\usepackage{amsthm}

\usepackage{graphicx}
  \graphicspath{{../images/}}
  \DeclareGraphicsExtensions{.pdf,.jpeg,.jpg,.png}

\newtheorem{definition}{Definition}[section]
\newtheorem{qstn}{Question}
\newtheorem{hyp}{Hypothesis}
\newtheorem{lem}{Lemma}[hyp]

% Document relocation and referencing
\usepackage{import}
\usepackage{nameref}
\usepackage{labelledrefs}


% *** ALIGNMENT PACKAGES ***
%
\usepackage{array}
\usepackage{adjustbox}
\usepackage{booktabs}
\usepackage{multirow}

\newcommand*\rot{\adjustbox{angle=90,lap=\width-(1em)}}% no optional argument here, please!

% *** SUBFIGURE PACKAGES ***
\usepackage[tight,footnotesize]{subfigure}
% subfigure.sty was written by Steven Douglas Cochran. This package makes it
% easy to put subfigures in your figures. e.g., "Figure 1a and 1b". For IEEE
% work, it is a good idea to load it with the tight package option to reduce
% the amount of white space around the subfigures. subfigure.sty is already
% installed on most LaTeX systems. The latest version and documentation can
% be obtained at:
% http://www.ctan.org/tex-archive/obsolete/macros/latex/contrib/subfigure/
% subfigure.sty has been superceeded by subfig.sty.

\hyphenation{op-tical net-works semi-conduc-tor}

% - Outer Joins
\def\ojoin{\setbox0=\hbox{$\Join$}%
  \rule{.25em}{.4pt}\llap{\rule[.46em]{.25em}{.4pt}}}
\def\leftouterjoin{\mathbin{\ojoin\mkern-8mu\Join}}
\def\rightouterjoin{\mathbin{\Join\mkern-8mu\ojoin}}
\def\fullouterjoin{\mathbin{\ojoin\mkern-8mu\Join\mkern-8mu\ojoin}}
