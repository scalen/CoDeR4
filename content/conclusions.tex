\begin{nestedsection}{Conclusions}{conclusions}
	We have presented Windowed Continuous Datalog (\emph{WC-Datalog}), a well-founded semantics for continuous reasoning at the expressivity of OWL 2 RL, extended incrementally from that of positive Datalog with temporal annotation, streams and windowing semantics.
	In this way it differs from other temporal interpretations of Datalog \citep{OrgunWadge92,Tuzhilin93} in that it forgoes the concepts of conventional temporal reasoning, replacing them with the windowing semantics of stream processing and casting a continuous program as a series of discrete states.
	WC-Datalog supports the derivation of sets of facts as the entailment of each discrete state of a continuous program and the derivation of concise streams of facts as the entailment of a continuous program over time, in a similar manner to the \emph{RStream}, \emph{IStream} and \emph{DStream} operators of the CQL continuous query language \citep{CQL}.
	In addition to this, however, it also supports the low-complexity derivation of the entailment of each discrete state from the streamed entailment of the continuous program via the application of windows to the concise streams, leading to the well-founded definition of a \emph{valid window} over the combined streams $P$ and $N$, being the outputs of the reasoning equivalents of the \emph{IStream} and \emph{DStream}.
	Thus WC-Datalog represents a well-founded semantics not only for continuous reasoning, but for coherent and concise streamed entailments of continuous programs, suitable for consumption in further stream processing.

	In order to express this semantics we have developed CoDeR, a processing model for Continuous Deductive Reasoning over RDF.
	This processing model is composed of a stream-based data model with a dual semantics and a stream-to-stream extension of a semantic interpretation of the relational algebra.
	The data model is composed of streams of RDF graphs annotated both with the time at which they became true and the time at which they will cease to be true, with the ordering of graphs by the former being the sequence of facts in $P$ and the ordering by the latter being the sequence of facts in $N$, hence providing a dual semantics for the stream.
	As such, the window inherent in a continuous program may be represented in the annotation of each graph in the streamed input, rather than ranged window operators in the algebra.
	Instead, the algebra makes use of \emph{valid window} operators to maintain the set of graphs that are both considered to be true at a given moment \emph{and} applicable at a given point in the processing.
	It also de-constructs the conventional window-join operator of stream-to-stream algebras for continuous querying into the union of the outputs of complementary State Modules (\emph{SteMs}) \citep{SteMs}, to which it is logically and functionally equivalent, thereby supporting greater variety in the possible plans that express a given continuous program.

	Finally, we have implemented the CoDeR model in the synchronous RDF stream-reasoning system R4.
	R4 utilises a variant of the Rete pattern matching algorithm \citep{forgy79} to produce processing plans for sets of rules expressed in RIF-Core \citep{w3crifcore}, the W3C recommendation for expressing sets of rules with the semantics of positive Datalog and the rule language used to express the set of entailment rules of OWL 2 RL \citep{w3cowl2profiles}.
	R4 is sound and complete with respect to the semantics of WC-Datalog by virtue of implementing CoDeR.
	Given that many state-of-the-art stream-querying and continuous-reasoning systems have varying semantics, and so the expected results of each system will vary for the same query definition, we contrasted the time efficiency and memory consumption of R4 with those of Sparkwave \citep{sparkwave} and Etalis \citep{EP-SPARQL}.
	We have demonstrated that R4 is comparable in both respects to these RDF stream-querying systems when applying low-range windows to high-throughput streams, despite Sparkwave and Etalis supporting more limited/less explicitly well-founded reasoning than that of R4.

	Moving forward, we plan to improve the time-efficiency of R4 by investigating more efficient join algorithms within its implementations of SteMs.
	At a higher level, we also plan to investigate alternate planing algorithms to improve the Quality of Service (\emph{QoS}) characteristics of the system, i.e. reduce the latency from receipt of some new graph to the production of entailments justified in part by that new graph.
	These primarily focus on the use of SteMs outside of the composition of window-joins, either taking an adaptive approach \citep{eddies,CACQ,CQELS} or utilising the wider array of static plans such an operator affords.
\end{nestedsection}