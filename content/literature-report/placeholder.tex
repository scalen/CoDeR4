\begin{nestedsection}{Related Work}{background}
	The research challenges stated in \refsec{intro-challenges} of this thesis span a number of different research areas, including the Semantic Web, deductive reasoning, multiple-query optimisation, stream processing, continuous querying and distributed systems.
	In every case, the challenges relate to drawing together knowledge from two or more of these areas and investigating the overlap therein.
	In this section I review the work present in the literature where it relates to my research challenges.
	
	The first point I discuss in \refsec{background-queries-vs-reasoning} is the implicit question that underpins the rest of the thesis: What are the similarities and differences between querying and reasoning?
	This defines to what extent I can apply research into forms of querying to this work on reasoning (as per \refdef{deductive-reasoning}).
	It also highlights the differing characteristics of querying and reasoning tasks, which informs my analysis of the literature in the later sections.

	I then explore the notion of stream processing as an answer to the \refcha{real-time}, and an answer to the ``continuous'' part of the \refcha{continuous-data}, in \refsec{background-native-stream-processing}.
	I focus my investigation on what characterises an algorithm that could be considered to perform reasoning over streamed data.
	One of the prime characteristics I identify in stream processing systems is a notion of ``forgetting'' older data in favour of that more recently received, epitomised by a data structure known as a ``sliding window''.
	The other important characteristic of such systems is an intrinsically incremental algorithm, which precludes the direct application of many techniques found in more powerful reasoning algorithms.
	This, in turn, highlights the \refcha{reasoning-expressivity} in conjunction with the \refcha{real-time}.

	Having elicited the key features of stream processing systems, I next detail the various methods for processing large amounts of data employed in the state of the art in \refsec{background-high-rate-volume}, focusing again on how those methods can be applied to stream processing in the face of the \refcha{data-rate}.
	Of particular interest are the ``divide and conquer'' technique and multi-query optimisation.
	The former refers to the practise of partitioning a problem and distributing it across many processing units, then combining the results into a complete solution.
	The latter involves detecting common tasks, known as ``common sub-expressions'', shared between a set of concurrent queries, then using that knowledge to eliminate duplicate task processing across the given query set.

	Finally, in \refsec{background-cont-changing-data} I explore the solutions proposed in the literature to the \refcha{continuous-data}.
	These solutions diverge from the concept of fixed plans explored in \refsec{background-queries-vs-reasoning}, exploring methods of adapting query plans dynamically in response to fluctuating data and operator statistics during processing.
	For a query plan to continuously adapt, the query planning algorithms used must be applied during stream processing;
	as such these planning algorithms must also display the characteristics common to stream processing algorithms as identified in \refsec{background-native-stream-processing}.

	In the concluding \refsec{background-conclusions}, I present a summary of the conclusions that I have drawn from literature.

	% \begin{table}[h]
	% 	\centering
	% 	\begin{tabular}{c | r | *{3}{c}}
	% 		\multicolumn{2}{c |}{} & \rot{Hello} & \rot{there} & \rot{World}\\\hline
	% 		& Hello & * & & \\
	% 		& there & & * & \\
	% 		\rot{\rlap{Greeting!}}
	% 		& World & & & *
	% 	\end{tabular}
	% 	\caption{Hello there, World!}
	% 	\labeltab{}
	% \end{table}

%	These elements can be divided into three primary concerns:
%	those of the Semantic Web as discussed in \refsec{background-semweb};
%	those of reasoning by logic programming and semantic networks as discusse in \refsec{background-reasoning};
%	and those of streaming as discussed in \refsec{background-streaming}, including both the processing of streamed data sources and the streaming of data between components of a distributed system.

%	In addition to these distinct concerns, the current state of the art concerning the overlap of these areas is discussed in \refsec{background-streaming-semantic-data}, and the diverse methods used to measure the performance of solutions within these research areas are discussed in \refsec{background-benchmarking}.

	\subimport{literature-report/}{queries-vs-reasoning.tex}
	\subimport{literature-report/}{native-stream-processing.tex}
	\subimport{literature-report/}{high-rate-volume.tex}
	\subimport{literature-report/}{cont-changing-data.tex}

%	\subimport{literature-report/}{semweb.tex}
%	\subimport{literature-report/}{reasoning.tex}
%	\subimport{literature-report/}{streaming.tex}
%	\subimport{literature-report/}{streaming-semantic-data.tex}
%	\subimport{literature-report/}{benchmarking.tex}
	
	\begin{nestedsection}{Preliminary Conclusions}{background-conclusions}
		
	\end{nestedsection}
\end{nestedsection}
