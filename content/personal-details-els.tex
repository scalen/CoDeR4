% Personal Details
\begin{frontmatter}
	\title{Continuous Deductive Reasoning over Streamed RDF Data}

	\author{David L. Monks\corref{cor1}}
	\ead{dm11g08@ecs.soton.ac.uk}
	\cortext[cor1]{Corresponding author.}
	\author{Rehab Albeladi}
	\ead{raab1g09@ecs.soton.ac.uk}
	\author{Nicholas Gibbins}
	\ead{nmg@ecs.soton.ac.uk}
	\address{
	          Electronics and Computer Science,\\
	          University of Southampton,\\
	          Southampton, SO17 1BJ,\\
	          United Kingdom
	}

	\begin{abstract}
		In this paper, we introduce Continuous Datalog, an extension of
conventional Datalog that adds the concept of streams of
temporally-annotated facts to conventional Datalog and defines the
notion of entailment over a windowed stream. We go on to define CoDeR,
an abstract model for continuous deductive reasoning that provides
both a data stream model based on Continuous Datalog and a minimal
algebra for expressing queries and programs.  Finally, we describe the
R4 stream reasoner, which implements the CoDeR model as a Rete-based
dataflow network and provide a brief comparative evaluation of R4
against ETALIS and Sparkwave.

	\end{abstract}

	\begin{keyword}
	      Deductive Reasoning\sep Streamed Data\sep Semantic Data\sep Continuous Processing\sep Plan Optimisation
	\end{keyword}

	\date{\today}

\end{frontmatter}
