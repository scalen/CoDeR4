\begin{nestedsection}{CoDeR Model for Continuous Deductive Reasoning}{model}
	CoDeR is an abstract model for continuous deductive reasoning over streamed RDF data that expresses the semantics of Continuous Datalog.
	It is composed of a stream-based data model for expressing the streams $S$, $P$ and $N$ of each ${CDP = \{A,S,R\} \vDash \{A',P,N\}}$, and a minimal algebra of stream-to-stream operators for expressing the persistent set of axiomatic rules $A$.
	It casts the problem of applying those rules to streamed RDF data as that of iterative pattern matching, in a manner inspired by the Rete pattern matching algorithm \citep{forgy79}, though it is meant to support any such algorithm rather than prescribing one itself.

	% Stream-based Data Model
	The data model of CoDeR is composed of two classes of streams.
	The first are \emph{input}/\emph{output} streams, and provide the semantics of streams $S$, $P$ and $N$ for any ${CDP}$ expressed by a CoDeR-based system;
	they are sequences of instances of RDF triples, where an instance of an RDF triple is an RDF triple annotated with the time interval for which it is valid \citep{SemanticStreamingManagement,sparkwave} according to the ${CDP}$ expressed: ${\langle (s,p,o),i_{e},i_{n} \rangle}$.
	This interval for any entailed fact instance is derived from the valid time intervals of the fact instances that justify it, as shown in \reffig{intersected-intervals}:
	\begin{figure}[t]
		\centering
		\includegraphics[width=0.45\textwidth]{intersected-intervals}
		\caption{Derivation of valid-time for entailed fact instance from those of its justifying fact instances.}
		\labelfig{intersected-intervals}
	\end{figure}
	if ${A = \{c\,\text{:-}\,a\,,\,b\,.\}}$ and ${W^{S,3}_{2} = \{ \langle a,1,3 \rangle , \langle b,2,4 \rangle \}}$, then ${CDP_{2} = A \cup W^{S,3}_{2} \vDash \{ \langle a,1,3 \rangle , \langle b,2,4 \rangle , \langle c,2,3 \rangle \}}$, where $a$, $b$ and $c$ are RDF triples.
	It is in the base case of this derivation that the semantic window range $R$ of a ${CDP}$ is expressed:
	each fact instance arriving on an \emph{input} stream has a valid interval from the time of its arrival at the system for the duration of the semantic window expressed by that range $R$.

	Given that the ordering of instances in the \emph{output} stream by the value of $i_{e}$ yields the sequence of instances in the semantic stream $P$ entailed by a ${CDP}$, and that their ordering by the value of $i_{n}$ yields the sequence of instances in the semantic stream $N$, this gives all streams of CoDeR a dual semantics with respect to Continuous Datalog.

	% Minimal Algebra

%	\subimport{conclusions/}{future-work.tex}
\end{nestedsection}