\begin{nestedsection}{Continuous Datalog}{semantics}
The first requirement of stream reasoning is a well-defined semantics.
This must take into account the continuous nature of a subset of the
base axioms of a stream reasoning problem, as well as its
instantaneous truth and entailments, in line with the work of the
reasoning community.  However, it must also provide a continuous
interpretation of those instantaneous entailments in order to
integrate with other stream-processing technologies.  More
specifically, an appropriate semantics for stream reasoning on the
Semantic Web must: allow the expression of OWL 2 ontologies to some
degree of expressivity; not sacrifice the precise semantics thereof
whilst incorporating the semantics of stream processing; and allow for
the continuous derivation of entailments in a concise manner.

	Continuous Datalog is the extension of positive Datalog with stream-processing semantics, with a view to accommodating the aims of the latter within the well-defined semantics of the former, which are shared with RIF-Core and therefore support reasoning at the expressivity of OWL 2 RL.
	This comprises the extension of Datalog programs to include sources of continuously generated/reported facts, and the application of sliding window semantics over those sources, thereby providing for the interpretation of Continuous Datalog programs as sequences of instantaneous states that are each valid positive Datalog programs.
	As such, these states each inherit the semantics of positive Datalog.
	Furthermore, it involves the continuous derivation of entailments from this sequence of instantaneous states as a set of streams of facts;
	these streams share a data model with the streams of continuous queries, as defined by the stream processing community, thereby allowing the integration of Continuous Datalog programs into stream processing pipelines.

In addition, it can be shown that these streams may be trivially
manipulated to retrieve the instantaneous sets of entailments produced
by the more semantically well-defined offerings from the reasoning
community.

	In regards to Continuous Datalog, an ``instant'', a ``stream of facts'' and a ``sliding window'' over a stream are defined as follows:

\begin{definition}[Instant]\label{def:continuous datalog: instant}
An instant is a discrete point in time, such that an instant ${i \in \mathbb{N}}$.

It should be noted that time is a continuous concept, with infinitely many divisions possible between any two discrete instants.
However, the resolution with which such divisions can be recognised will, in practise, have a lower bound set by hardware limitations, or may be defined as part of some specific implementation.
In these cases, anything that occurs between discrete instants is considered to have occurred at the next recognised instant.
\end{definition}

\begin{definition}[Fact Instance]\label{def:continuous datalog: fact instance}
A fact instance is a combination of the signature of a fact (its
predicate $p$, subject $s$ and object $o$) and its entailment time
$t$; a fact instance is denoted ${p(s,o):t}$, and is unique.
\end{definition}

\begin{definition}[Stream of Facts]\label{def:continuous datalog: stream}
A stream of facts is a \emph{sequence} of Datalog facts ordered by
some temporal characteristic of each instance of each fact, e.g. the
time at which each was entailment by some process at the source of the
stream.  The distinction between instances of facts within a
stream is represented in Axiom~\ref{axiom:continuous datalog: window disjointness}.
\end{definition}

\begin{definition}[Sliding Window]\label{def:continuous datalog: window}
A sliding window of range ${X \in \mathbb{N}}$ over any stream $S$ at
any given instant ${i \in \mathbb{N}}$ is the \emph{set} of instances
of facts that occur in $S$ during the interval ${(i-X,i]}$, and is
  denoted ${W^{S,X}_{i}}$.
\end{definition}

The following axioms define the semantics of streams and windows over
those streams in Continuous Datalog:

\begin{axiom}\label{axiom:continuous datalog: window range leq 0}
A window of range $0$ over any stream $S$ at any instant 
${i \in \mathbb{N}}$ is always empty.
\begin{equation*}
W^{S,0}_{i} = \varnothing
\end{equation*}
\end{axiom}

\begin{axiom}\label{axiom:continuous datalog: window composition}
A window of any range ${X \ge 0}$ over any stream $S$ at any
given instant ${i \in \mathbb{N}}$ is equal to the union of the
$X$ consecutive windows of range 1 over the stream $S$ preceding and
including that at instant $i$.
\begin{equation*}
W^{S,X}_{i} = \mathop{\cup}_{i=0}^{X-1} W^{S,1}_{i-i}
\end{equation*}
\end{axiom}

\begin{axiom}\label{axiom:continuous datalog: window disjointness}
A window of any range ${X \in \mathbb{N}}$ over any stream $S$ at any
instant ${i \in \mathbb{N}}$ is disjoint from a second window of any
range ${Y \in \mathbb{N}}$ over $S$ at any instant after ${i - X}$.
\begin{equation*}
\forall X' \geq X \, \left( W^{S,X}_{i} \cap W^{S,Y}_{i-X'} = \varnothing \right)
\end{equation*}
\end{axiom}

	Traditionally, a set of axioms $A$ entails another set of axioms $E$ in which the existence of one fact in $E$ may be justified by many distinct subsets of $A$.
	When considering streams of facts, in which one fact may occur multiple times and be considered distinct based on its entailment time, temporal information would be lost if sets of axioms including fact instances (denoted $A_{inst}$) were to entail sets of axioms without.
	As such, where a fact would be entailed by some set of facts $A$ that are instantiated on some stream, an instance of that fact should be entailed by the set of instances of those facts present on that stream.
	Furthermore, each such justification in $A_{inst}$ for a fact entails a unique instance of that fact, such that each entailed instance has a single justification within the original set of axioms, and many instances of the entailed fact may occur in the entailed set of axioms $E_{inst}$.
	This form of entailment shall be henceforth referred to as ``instance entailment'', and shall be the norm when discussing the entailments of sets of axioms that include instances of facts.
	
	It should be noted that removing all temporal information from a set of axioms including fact instances $A_{inst}$ produces the set of traditional Datalog axioms $A$ instantiated in $A_{inst}$.
	This is be represented as the function ${trad\left( A_{inst} \right) = A}$.
	In addition, applying the ${trad()}$ function to both the original axiom instances $A_{inst}$ and the entailed axiom instances $E_{inst}$ maintains the relationship of entailment between the two sets:
	\[ \left( A_{inst} \vDash E_{inst} \right) \rightarrow \left( A \vDash E \right) \]

	From this description of \emph{instance entailment}, Axiom~\ref{axiom:continuous datalog: entailment disjointness} may be derived.

	\begin{axiom}\label{axiom:continuous datalog: entailment disjointness}
		If two sets, $A_{inst}$ and $B_{inst}$, of axiom instances are disjoint, then the any set $E_{inst}$ of axiom instances each justified at least in part by $A_{inst}$ is necessarily disjoint from any set $F_{inst}$ of axiom instances each justified at least in part by $B_{inst}$, because entailed fact instances are distinct with regards to the fact instances that justify them.
		\begin{multline*}
			\left( A_{inst} \cap B_{inst} = \varnothing \right) \wedge \left( A_{inst} \vDash E_{inst} \right) \\
				\wedge \left( B_{inst} \vDash F_{inst} \right) \rightarrow \left( E_{inst} \cap F_{inst} = \varnothing \right)
		\end{multline*}
	\end{axiom}

	Given these definitions, a Continuous Datalog program may be formally defined, both in terms of streams and in terms of its instantaneous states:

	\begin{definition}[Continuous Datalog Program]\label{def:continuous datalog: CDP}
		A Continuous Datalog Program (${CDP}$) is composed of a set of Positive Datalog axioms $A$, a set of streams of facts $S$ and the range ${R \in \mathbb{N}}$ of the windows over each stream in $S$.
		This specifies that the streamed data from $S$ is to be considered valid from its arrival for the duration represented by $R$.
		Such a continuous program entails a triple composed of the traditional entailments of ${A \vDash A'}$, a stream of entailed instances $P$ and a stream of instance negations $N$, whose contents are defined in Axioms~\ref{axiom:continuous datalog: positive window increment},~\ref{axiom:continuous datalog: negative window increment} and~\ref{axiom:continuous datalog: entailment precedes negation}.
		Streams $P$ and $N$ are therefore the continuous reasoning equivalents of the streams produced by the \emph{IStream} and \emph{DStream} operators of the \emph{CQL} continuous query language \citep{CQL} respectively.
		The truth value of a ${CDP}$ is instantaneous, with the the truth function of Continuous Datalog $T^{CD}$ taking both a program ${CDP}$ and an instant ${i \in \mathbb{N}}$ as arguments.
		\begin{equation*}
			CDP = \{A,S,R\} \vDash \{A',P,N\}
		\end{equation*}
	\end{definition}

	\begin{definition}[State of a ${CDP}$]\label{def:continuous datalog: CDPt}
		The state of a ${CDP}$ at a given instant ${i \in \mathbb{N}}$ (${CDP_{i}}$) is the union of the axioms $A$ and the window $W^{S,R}_{i}$.
		As both $A$ and ${trad\left( W^{S,R}_{i} \right)}$ are sets of Datalog axioms, a ${CDP_{i}}$ constitutes a valid Datalog program that may be evaluated by the truth function of Datalog $T^{D}$, the result of which is the definition for the instantaneous truth of the ${CDP}$ at instant $i$ by $T^{CD}$.
		The instantaneous entailments of a $CDP_{i}$ are the union of the persistent entailments of ${A \vDash A'}$ (which are guaranteed to be entailed by every state, as Positive Datalog is monotonic) and the set $E_{i}$ of instances of the transient entailments that are justified at least in part by some subset of $W^{S,R}_{i}$.
		\begin{equation*}
			T^{CD} \left( CDP, i \right) = T^{D} \left( CDP_{i} \right)
		\end{equation*}
		\begin{equation*}
			CDP_{i} = A \cup W^{S,R}_{i} \vDash A' \cup E_{i}
		\end{equation*}
	\end{definition}


	The following axioms define the semantics of the streams of entailments $P$ and and negations $N$ of a ${CDP}$ in terms of the instantaneous entailments of the sequence of states of that continuous program.

	\begin{axiom}\label{axiom:continuous datalog: positive window increment}
		The window of range 1 at any given instant ${i \in \mathbb{N}}$ over the stream $P$ entailed by a ${CDP}$ is defined as the set of facts that are entailed by the state of the ${CDP}$ at instant $i$ that were not entailed by the state of ${CDP}$ at the previous instant instance ${i-1}$.
		\begin{equation*}
			W^{P,1}_{i} = \left( A' \cup E_{i} \right) \setminus \left( A' \cup E_{i-1} \right) = E_{i} \setminus E_{i-1}
		\end{equation*}
	\end{axiom}

	\begin{axiom}\label{axiom:continuous datalog: negative window increment}
		The window of range 1 at any given instant ${i \in \mathbb{N}}$ over the stream $N$ entailed by a ${CDP}$ is defined as the set of facts that were entailed by the state of the ${CDP}$ at the previous instant instance ${i-1}$ that are no longer entailed by the state of ${CDP}$ at instant $i$.
		\begin{equation*}
			W^{N,1}_{i} = \left( A' \cup E_{i-1} \right) \setminus \left( A' \cup E_{i} \right) = E_{i-1} \setminus E_{i}
		\end{equation*}
	\end{axiom}

	\begin{axiom}\label{axiom:continuous datalog: entailment precedes negation}
		Facts may only be negated by some part of their justification leaving the sliding window $W^{S,R}_{i}$, this necessarily being after they were entailed.
		As such, a window of any range ${X \in \mathbb{N}}$ over the stream of entailments $P$ at any given instant ${i \in \mathbb{N}}$ is disjoint from a window of any range ${Y \in \mathbb{N}}$ over the the stream of entailment negations $N$ at any instant after ${i - X}$
		\begin{equation*}
			\forall X' \geq X \left( W^{P,X}_{i} \cap W^{N,Y}_{i-X'} = \varnothing \right)
		\end{equation*}
	\end{axiom}

	From this, we can see that each instance of an entailment in some $E_{i}$ may take the same form as an instance of any other fact in a stream of facts, being ${p(s,o):t}$, where $t$ is the real time associated with the instant $i_{e}$ in which ${p(s,o)}$ becomes newly justified by the ${CDP}$
	(more specifically, by some subset of the instances in $W^{S,R}_{i_{e}}$).
	As such, the stream of entailment instances $P$ of a ${CDP}$ takes the same form as the \emph{streams of facts} already described, thus producing a concise stream of entailments that may be read by further stream processing.
	In addition, the stream of negations $N$ of a ${CDP}$ also takes this form, but differs from the streams discussed thus far in that it is not ordered by the entailment time $i_{e}$ of the entailment instances but by the time at which their justification \emph{ceases} to be valid
	(i.e. the first instant ${i_{n} \geq i_{e}}$ in which $W^{S,R}_{i_{n}}$ no longer contains the subset of fact instances first present in $W^{S,R}_{i_{e}}$ that justified the entailment instance ${p(s,o):t}$).

	Thus far the translation from instantaneous sets of entailments to streams of entailments has been stated axiomatically as the definition of those streams.
	However, it can be shown from these axioms,
	as in \ref{sec:semantics: proof for entailments},
	that it is also possible to reconstruct exactly the instantaneous set $E_{i}$ of entailment instances of any ${CDP}$ at any given instant from the windows of sufficient range over its entailed streams $P$ and $N$ at that instant.
	\[ \forall Y \geq X \geq R \, \left( E_{i} = W^{P,X}_{i} \setminus W^{N,Y}_{i} \right) \]
	The minimum range $X$ is such to achieve the \emph{complete} set of entailment instances $E_{i}$ of the ${CDP}_{i}$.
	In addition, the minimum range $Y$ is such to achieve the \emph{correct} set $E_{i}$.
	As such, a \emph{valid window} that contains only and exactly those instances of entailments in $E_{i}$ at every instant $i$ may be trivially maintained from the streams $P$ and $N$ continuously entailed by some ${CDP}$.
\end{nestedsection}
