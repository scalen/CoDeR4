\begin{nestedsection}{Continuous Datalog}{semantics}
The first requirement of stream reasoning is a well-defined semantics.
that takes into account both the continuous nature of a subset of the
base axioms of a stream reasoning problem, as well as its
instantaneous truth and entailments, in line with the work of the
reasoning community.  However, it must also provide a continuous
interpretation of those instantaneous entailments in order to
integrate with other stream-processing technologies.  More
specifically, an appropriate semantics for stream reasoning on the
Semantic Web must: allow the expression of OWL 2 ontologies to some
degree of expressivity; not sacrifice the precise semantics thereof
whilst incorporating the semantics of stream processing; and allow for
the continuous derivation of entailments in a concise manner.

Continuous Datalog is an extension of positive Datalog with
stream-processing semantics, with a view to accommodating the aims of
the latter within the well-defined model-theoretic semantics of the
former, which are shared with RIF-Core \cite{w3crifbld} and therefore
support reasoning at the expressivity of OWL 2 RL.  This comprises the
extension of Datalog programs to include sources of continuously
generated/reported facts, and the application of sliding window
semantics over those sources, thereby providing for the interpretation
of Continuous Datalog programs as sequences of instantaneous states
that are each valid positive Datalog programs.  As such, these states
each inherit the semantics of positive Datalog.  Furthermore, it
involves the continuous derivation of entailments from this sequence
of instantaneous states as a set of streams of facts; these streams
share a data model with the streams of continuous queries, as defined
by the stream processing community, thereby allowing the integration
of Continuous Datalog programs into stream processing pipelines.

In addition, it can be shown that these streams may be trivially
manipulated to retrieve the instantaneous sets of entailments produced
by the more semantically well-defined offerings from the reasoning
community.

%% fixme
%	In regards to Continuous Datalog, an ``instant'', a ``stream of facts'' and a ``sliding window'' over a stream are defined as follows:

\begin{definition}[Datalog]
\labeldef{continuous datalog: datalog program}

A conventional Datalog program $\Pi = \langle R, F\rangle$ comprises a
set of clauses, divided into a finite set of rules $R$ and a finite
set of ground facts $F$. Facts are formulae of first-order logic, and
rules are of the form:

\[\alpha_0 :\!-\ \alpha_1 , \ldots , \alpha_n\]
where $\alpha_i$ are atomic formulae.

The Herbrand base $\hat{H}$ of $\Pi$ is the set of all facts that can
be expressed using the predicates in $\Pi$; for a program $\Pi =
\langle R, F \rangle$, $F \subseteq \hat{H}$. Note that we do not
distinguish between the predicates in $F$ and those in the heads of
the rules in $R$ (as extensional and intensional predicates), as is
the case in the model theory for Datalog presented in
\cite{datalog-basics}; in this respect, our treatment follows that of
the model theory for Prolog presented in \cite{prolog-semantics}. A
Herbrand interpretation $\mathcal{I}$ of $\Pi$ is any subset of
$\hat{H}$, and a Herbrand interpretation $\mathcal{I}$ that satisfies
the set of clauses in program $\Pi$ is a Herbrand model of $\Pi$,
written $\models_{\mathcal{I}} \Pi$.

We define the concept of entailment in Datalog as follows. A fact $f$
is entailed by a program $\Pi$ iff each interpretation satisfying
$\Pi$ (satisfying every clause in $\Pi$) also satisfies $f$, written
$\Pi \models f$. The entailment of $\Pi$ is therefore the set of all
facts that logically follow from $\Pi$:

\[cons(\Pi) = \{ f \in \hat{H} | \Pi \models f \}\]
\end{definition}

\begin{definition}[Justification]
\labeldef{continuous datalog: justification}
A {\em justification} for the entailment $E$ of some Datalog program
${\langle R , F \rangle \models E}$ is a minimal set
${J \subseteq F}$ such that ${\langle R , J \rangle \models E}$.
\end{definition}

\begin{definition}[Instant]
\labeldef{continuous datalog: instant}
We assume a discrete, ordered time domain $\mathbb{T}$; an instant $i
\in \mathbb{T}$ is a point in time. It should be noted that time is a
continuous concept, with infinitely many divisions possible between
any two discrete instants.  However, the resolution with which such
divisions can be recognised will, in practise, have a lower bound set
by hardware limitations, or may be defined as part of some specific
implementation.  In these cases, anything that occurs between discrete
instants is considered to have occurred at the next recognised
instant.
\end{definition}

\begin{definition}[Fact Instance]
\labeldef{continuous datalog: fact instance}
A {\em fact instance} is a fact with some additional temporal
annotation; it is a tuple containing a fact $f$ and the timestamp
$t \in \mathbb{T}$ at which $f$ became true, written
$\langle f, t \rangle$. Two fact instances containing the same fact
and timestamp are considered identical.

We represent the relationship between a fact (without temporal
annotation) and a fact instance (with temporal annotation) with the
function ${strip : \hat{H} \times \mathbb{T} \rightarrow \hat{H}}$
which maps a fact instance to its underlying fact.
\end{definition}

\begin{definition}[Stream of Facts]
\labeldef{continuous datalog: stream}
A stream of facts $S$ is a \emph{sequence} of fact instances ordered
by their timestamps; in conjunction with the distinction between
multiple instances of a fact by their timestamp, this definition justifies
Axiom~\ref{axiom:continuous datalog: window disjointness}.
\end{definition}

\begin{definition}[Sliding Window]
\labeldef{continuous datalog: window}

A sliding window of range ${x \in \mathbb{N}}$ over any stream $S$ at
any given instant ${i \in \mathbb{N}}$ is the \emph{set} of fact
instances that occur in $S$ during the interval ${(i-x,i]}$, written
${W^{S,x}_{i}}$.

\[ W^{s,x}_i = \{ \langle f, t \rangle | \langle f, t \rangle \in S \land (i-x) < t \leqslant i \}\]  

\end{definition}

The following axioms define the semantics of streams and windows over
those streams in Continuous Datalog:

\begin{axiom}\label{axiom:continuous datalog: window range leq 0}
A window of range $0$ over a stream $S$ at an instant 
${i \in \mathbb{N}}$ is always empty.
\begin{equation*}
W^{S,0}_{i} = \varnothing
\end{equation*}
\end{axiom}

\begin{axiom}\label{axiom:continuous datalog: window composition}
A window of range ${x \ge 0}$ over a stream $S$ at an instant 
${i \in \mathbb{N}}$ is equal to the union of the $x$ consecutive
windows of range $1$ over the stream $S$ preceding and including that at
instant $i$.
\begin{equation*}
W^{S,x}_{i} = \mathop{\cup}_{j=0}^{x-1} W^{S,1}_{i-j}
\end{equation*}
\end{axiom}

\begin{axiom}\label{axiom:continuous datalog: window disjointness}
A window of range ${x \in \mathbb{N}}$ over a stream $S$ at an
instant ${i \in \mathbb{N}}$ is disjoint from a second window of 
range ${y \in \mathbb{N}}$ over $S$ at any instant before $i - x$.
\begin{equation*}
W^{S,x}_{i} \cap W^{S,y}_{i-z} = \varnothing
\end{equation*}
where $z \geq x$.
\end{axiom}

In \refdef{continuous datalog: datalog program}, we took a
Datalog program to consist of a set of rules and a set of facts.
Taken together with the set of rules $R$, the set of facts $F$ entails
another set of facts $E$ in which the existence of one fact in $E$ may
be justified by many distinct subsets of $F$.

In order to extend this
definition of Datalog programs to incorporate fact insatnces,
as in \refdef{continuous datalog: fact instance}, we define a
time-annotated Datalog program in \refdef{continuous datalog: ta datalog program}.

\begin{definition}[Time-Annotated Datalog Program]
\labeldef{continuous datalog: ta datalog program}
A time-annotated Datalog program $\Pi^T = \langle R, F^T\rangle$ is a tuple
which comprises a finite set of rules $R$ and a finite set of fact instances
$F^T$. Fact instances are extensional atomic formulae of first-order logic
annotated with a timestamp, as in \refdef{continuous datalog: fact instance},
and rules are of the same form as those of conventional Datalog programs.

As discussed in \refdef{continuous datalog: datalog program,continuous datalog: fact instance},
in the context of the Semantic Web we treat fact instances as time-annotated
3-tuples ${\langle s, p, o \rangle | t}$.
\end{definition}

When considering a time-annotated Datalog program, in which there may be
many distinct instances of a given fact, temporal information regarding when
instances of that fact became true would be lost if such a program were to entail
sets of plain facts. As such, where a fact $f$ would be entailed by some Datalog
program ${\Pi = \langle R, \{ strip(x) : x \in W^{S,x}_i \} \rangle}$,
for some window over some stream $S$, an instance ${f | t}$ of $f$ should be
entailed by the time-annotated Datalog program
${\Pi^T = \langle R, W^{S,x}_i \rangle \vdash f | t}$.

It follows that applying the ${strip()}$ function to all the fact instances in
both the set of fact instances $F^T$ in some ${\Pi^T = \langle R, F^T \rangle \vDash E^T}$
and the set of entailed fact instances $E^T$ maintains the relationship of entailment:
\begin{multline*}
	\left( \langle R, F^T \rangle \vDash E^T \right) \rightarrow \\
		\left( \langle R, \{ strip(x) : x \in F^T \} \rangle \vDash \{ strip(x) : x \in E^T \} \right)
\end{multline*}

It should be noted that \refdef{continuous datalog: justification} of entailment
justification is equally applicable to time-annotated Datalog programs,
simply replacing entailed and justifying facts with fact instances.
Furthermore, each justification for $f$ in some $F^T$ entails a unique
instance ${f | t}$, such that many instances of $f$ may occur in the set
$E^T$ of fact instances entailed by
${\Pi^T = \langle R, W^{S,x}_i \rangle}$, each with a single
justification.  As such, entailment justification for fact instances may be
defined formally by \refdef{continuous datalog: instance justification}.
\begin{definition}[Instance Entailment Justification]
\labeldef{continuous datalog: instance justification}
	A justification for some entailment ${f|t}$ of some time-annotated Datalog program
	${\Pi^T = \langle R , F^T \rangle \vdash f|t}$ is the single minimal set ${J^T \subseteq F^T}$
	such that ${\Pi^{T\prime} = \langle R , J^T \rangle \vdash f|t}$.
	The mapping of the fact instance ${f | t}$ to the minimal subset $J^T$
	within ${\Pi^T = \langle R, F^T \rangle}$ that justifies it is expressed
	as the function ${just : \mathbb{\Pi^T} , \mathbb{F^T} \rightarrow \{ \mathbb{F^T} \}}$.
\end{definition}

This allows us to formally specify that the time $t$ at which an entailed fact
instance ${f | t}$ becomes true is the last time at which any of its
justifying instances become true:
\[ t = max(\{ t^\prime : f^\prime | t^\prime \in just(\Pi^T, f | t) \}) \]
As such, the set of timestamps $T$ for some set $E^T$ of instances entailed by some
program ${\langle R, F^T \rangle}$ is a subset of the timestamps for the instances in $F^T$.

From this description of instance entailment,
Axiom~\ref{axiom:continuous datalog: entailment disjointness} may be
derived:

\begin{axiom}\label{axiom:continuous datalog: entailment disjointness}
If two sliding windows over the same stream $S$ cover non-overlapping time periods,
then a set ${E^T1}$ of fact instances entailed by ${\langle R, W^{S,x}_i \rangle}$
is necessarily disjoint from any set ${E^T2}$ of fact instances entailed by
${\langle R, W^{S,y}_{i-z} \rangle}$, where ${z \geq x}$.
% This is because the timestamps of the instances ${E^T1}$ are a subset of those of
% the instances $W^{S,x}_i$, the timestamps of the instances ${E^T2}$ are a subset
% of those of the instances $W^{S,y}_{i-z}$, fact instances are distinct by timestamp and
% ${\{ t : f | t \in W^{S,x}_i\} \cap \{ t : f | t \in W^{S,y}_{i-z}\} = \varnothing}$.
\[ \{ e : \langle R , W^{S,x}_i \rangle \vdash e \} \cap \{ e : \langle R , W^{S,y}_{i-z} \rangle \vdash e \} = \varnothing \]
\end{axiom}

Given these definitions, a Continuous Datalog program may be formally
defined, both in terms of streams and in terms of its instantaneous
states:

\begin{definition}[Continuous Datalog Program]\label{def:continuous datalog: CDP}
A Continuous Datalog Program $\Pi^C$ is a tuple ${\langle R, F, S, r
\rangle}$ which comprises a finite set of rules $R$, a finite set of persistent
facts $F$, a stream of facts $S$ and a range ${r \in \mathbb{N}}$ for
the sliding window over $S$. This specifies that the streamed instances from $S$ are
to be considered valid from their timestamp for the duration represented
by $r$.

The entailment of such a program is composed of a stream of entailed
instances $P$ and a stream of instance negations $N$, whose contents
are defined in 
Axioms~\ref{axiom:continuous datalog: positive window increment}, 
\ref{axiom:continuous datalog: negative window increment} and
\ref{axiom:continuous datalog: entailment precedes negation}. 
These specify that the streams $P$ and $N$ are the continuous
reasoning equivalents of the streams produced by the \emph{IStream}
and \emph{DStream} operators of the \emph{CQL} continuous query
language \citep{CQL}, respectively.
\[ \Pi^C = \langle R, F, S, r \rangle \vDash \langle E, P, N \rangle \]
The truth value of a $\Pi^C$ is instantaneous, with the the truth
function of Continuous Datalog $T^{CD}$ taking both a program ${\Pi^C}$
and an instant ${i \in \mathbb{N}}$ as arguments.
\end{definition}

\begin{definition}[State of a $\Pi^C$]\label{def:continuous datalog: CDPt}
The state of a $\Pi^C$ at a given instant $i$ is the time-annotated
Datalog program ${\Pi^C_i = \langle R, F^0 \cup W^{S,r}_{i} \rangle}$,
where ${F^0 = \{ f | 0 : f \in F \}}$ is the set of persistent fact instances produced by
annotating each fact in $f$ in $F$ as becoming true at the first recognised instant.

As both $F$ and ${W_i = \{ strip(x) : x \in W^{S,r}_{i} \}}$ are sets of conventional
Datalog facts, the tuple ${\langle R, F \cup W_i}$
constitutes a valid Datalog program that may be evaluated by the truth
function of Datalog $T^{D}$, the result of which is the definition for the
instantaneous truth of $\Pi^C$ at instant $i$ by $T^{CD}$.
\begin{multline*}
T^{CD} \left( \langle R, F, S, r \rangle, i \right) = \\
	T^{D} \left( \langle R, F \cup \{ strip(x) : x \in W^{S,r}_{i} \} \right)
\end{multline*}

The instantaneous entailments of a $\Pi^C_i$ are the union of the set of
persistent entailments of ${\langle R, F^0 \rangle \vDash E^0}$ (which are
guaranteed to be entailed by every state, as positive Datalog is monotonic),
and the set $E^T_{i}$ of transient entailments whose justifications are at
least partially contained in $W^{S,r}_i$.
\begin{equation*}
\Pi^C_i = \langle R, F^0 \cup W^{S,r}_{i} \rangle \vDash E^0 \cup E^T_{i}
\end{equation*}
$E^T_i$ is defined more formally as
${E^T_{i} = \{ x : \left( \Pi^C_i \vdash x \right) \wedge
\left( just(\Pi^C_i,x) \cap W^{S,r}_i \neq \varnothing \right) \}}$.
By this definition and Axiom~\ref{axiom:continuous datalog: entailment disjointness},
$E^0$ is guaranteed to be disjoint from every $E^T_i$
for every state $\Pi^C_i$ of every Continuous Datalog Program $\Pi^C$.
\end{definition}

The following axioms define the semantics of the streams of
entailments $P$ and and negations $N$ of a $\Pi^C$ in terms of the
instantaneous entailments of the sequence of states of that continuous
program.

\begin{axiom}
\label{axiom:continuous datalog: positive window increment}
The window of range 1 at any given instant ${i \in \mathbb{N}}$ over
the stream $P$ entailed by a $\Pi^C$ is defined as the set of fact instances
that are entailed by the state of the $\Pi^C$ at instant $i$ that were
not entailed by the state of $\Pi^C$ at the previous instant ${i-1}$.
\begin{equation*}
W^{P,1}_{i} = \left( E^0 \cup E^T_{i} \right) \setminus
\left( E^0 \cup E^T_{i-1} \right) = E^T_{i} \setminus E^T_{i-1}
\end{equation*}
\end{axiom}
\begin{axiom}
\label{axiom:continuous datalog: negative window increment}
The window of range 1 at any given instant ${i \in \mathbb{N}}$ over
the stream $N$ entailed by a $\Pi^C$ is defined as the set of fact instances
that were entailed by the state of the $\Pi^C$ at the previous instant
${i-1}$ that are no longer entailed by the state of $\Pi^C$ at instant $i$.
\begin{equation*}
W^{N,1}_{i} = \left( E^0 \cup E^T_{i-1} \right) \setminus
\left( E^0 \cup E^T_{i} \right) = E^T_{i-1} \setminus E^T_{i}
\end{equation*}
\end{axiom}

\begin{axiom}
\label{axiom:continuous datalog: entailment precedes negation}
Fact instances may only be negated by some part of their justification leaving
the sliding window $W^{S,r}_{i}$, this necessarily being after they
were entailed.  As such, a window of any range ${x \in \mathbb{N}}$
over the stream of entailments $P$ at any given instant ${i \in \mathbb{N}}$
is disjoint from a window of any range ${y \in \mathbb{N}}$ over the
stream of entailment negations $N$ at any instant before ${i - x}$.
\begin{equation*}
W^{P,x}_{i} \cap W^{N,y}_{i-z} = \varnothing
\end{equation*}
where $z \geq x$.
\end{axiom}

From this, we can see that each entailed fact instance ${f|t}$ in some
$E^T_{i}$, of the form ${f|t = \langle s, p, o \rangle | t}$, has a
timestamp $t$ that is the real-time associated with the state
$\Pi^C_{i_{e}}$ in which its particular justification for ${f|t}$ is
first complete. As such, the stream of entailed fact instances $P$ of
a $\Pi^C$ takes precisely the same form as the streams of facts already
described, thus producing a concise stream of entailments that may be
read by further stream processing.  In addition, the stream of negations
$N$ of a $\Pi^C$ also takes this form, but differs from the streams
discussed thus far in that it is not ordered by the entailment/assertion
time of the entailed fact instances, but by the real-time associated
with the first state $\Pi^C_{i_{n}}$ in which their justification
\emph{ceases} to be complete, i.e. ${just(\Pi^C_{i_{e}}, f|t) \subseteq W^{S,r}_{i_n - 1}}$
but ${just(\Pi^C_{i_{e}}, f|t) \not\subseteq W^{S,r}_{i_n}}$.

Thus far the translation from instantaneous sets of entailments to
streams of entailments has been stated axiomatically as the definition
of those streams.  However, it can be shown from these axioms, as in
\ref{sec:semantics: proof for entailments}, that it is also possible
to reconstruct exactly the instantaneous set $E^T_{i}$, of instances
entailed at any given instant $i$ by any $\Pi^C = \{ R, F, S, r \}$,
from the sliding windows of sufficient range over its entailed streams
$P$ and $N$ at $i$:
\[ E^T_{i} = W^{P,x}_{i} \setminus W^{N,y}_{i} \]
where ${y \geq x \geq r}$. The minimum range $x$ is such to achieve
the \emph{complete} set of entailment instances $E^T_{i}$ of the
$\Pi^C_{i}$. In addition, the minimum range $y$ is such to achieve the
\emph{correct} set $E^T_{i}$. As such, a \emph{valid window} that
contains only and exactly those entailed instances $E^T_{i}$ at every
instant $i$ may be trivially maintained from the streams $P$ and $N$
continuously entailed by some $\Pi^C$.

To sum up, a time-annotated Datalog program
${\Pi^C_i = \langle R, F^0 \cup W^{S,r}_i \rangle}$ may be derived from
a continuous Datalog program ${\Pi^C = \langle R, F, S, r \rangle}$ by
applying a sliding window $W^{S,r}_i$ of range $r$ over the stream $S$ at
instant $i$ and instantiating the facts in $F$ as $F^0$. At the same time,
the entailments of a ${\Pi^C_i \vDash E^0 \cup E^T_i}$ may be derived from
the entailments of the ${\Pi^C \vDash \langle E, P, N \rangle}$ as the union
of the entailments of ${\langle R, F^0 \rangle \vDash E^0}$ and the set
difference of the windows $W^{P,r}_i$ and $W^{N,r}_i$ of range $r$ over
streams $P$ and $N$ at instant $i$.
Furthermore, a conventional Datalog program ${\Pi = \langle R, F \rangle}$
may be derived from a time-annotated Datalog program $\Pi^C_i$ by
${strip}$ping the time annotations from the set of facts
${F^0 \cup W^{S,r}_i}$, as the entailments of ${\Pi \vDash E}$ may be
derived from the entailments of $\Pi^C_i$ by $strip$ping them.
This yields the commutative diagram in \reffig{CD-TD-D}.
\begin{figure}
	\centering
	$
	\begin{CD}
		{\Pi^C = \langle R, F, S, r \rangle} @>\vDash>> {\langle E, P, N \rangle} \\
		@V{window}VV @VV{window}V \\
		{\Pi^C_i = \langle R, F^0 \cup W^{S,r}_i \rangle} @>\vDash>> {E^0 \cup E^T_i} \\
		@V{strip}VV @VV{strip}V \\
		{\Pi = \langle R, F \cup W_i \rangle} @>\vDash>> {E \cup E_i}
	\end{CD}
	$
	\caption{Commutative diagram from Continuous Datalog programs to Datalog entailments.}
	\labelfig{CD-TD-D}
\end{figure}
\end{nestedsection}
