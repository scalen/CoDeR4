\begin{nestedsection}{Semantics: Windowed Continuous Datalog}{semantics}
The first requirement of stream reasoning is a well-defined semantics
that takes into account both the continuous nature of a subset of the
base axioms of a stream reasoning problem, as well as its
instantaneous truth and entailments, in line with the work of the
reasoning community.  However, it must also provide a continuous
interpretation of those instantaneous entailments in order to
integrate with other stream-processing technologies.  More
specifically, an appropriate semantics for stream reasoning on the
Semantic Web must: allow the expression of OWL 2 ontologies to some
degree of expressivity; not sacrifice the precise semantics thereof
whilst incorporating the semantics of stream processing; and allow for
the continuous derivation of entailments in a concise manner.

Windowed Continuous Datalog (henceforth WC-Datalog) is an extension of positive
Datalog with stream-processing semantics, with a view to accommodating
the aims of the latter within the well-defined model-theoretic
semantics of the former, which are shared with RIF-Core
\cite{w3crifbld} and therefore support reasoning at the expressivity
of OWL 2 RL. As part of its application to Semantic Web reasoning, we
assume that the atomic formulae of the Datalog programs described
below correspond to RDF triples: $\langle subj, pred, obj\rangle$.

WC-Datalog differs from conventional Datalog through the addition of
sources of continuously generated or reported facts (i.e streams), and
the application of windowing semantics over those sources,
thereby providing for the interpretation of WC-Datalog programs as
sequences of instantaneous states that may each be reduced to valid positive Datalog
programs.  As such, these states each inherit the semantics of
positive Datalog.  Furthermore, it involves the continuous derivation
of entailments from this sequence of instantaneous states as a stream
of facts; these streams share a data model with the streams of
continuous queries, as defined by the stream processing community,
thereby allowing the integration of WC-Datalog programs into stream
processing pipelines. Finally, we show that these streams may be
manipulated to retrieve the instantaneous sets of entailments produced
by the equivalent Datalog programs.

\begin{definition}[Datalog]
\labeldef{continuous datalog: datalog program}

A conventional Datalog program $\Pi = \langle R, F\rangle$ comprises a
set of clauses, divided into a finite set of rules $R$ and a finite
set of ground facts $F$. Facts are formulae of first-order logic, and
rules are of the form:

\[ \alpha_0 \Leftarrow \alpha_1 \land \ldots \land \alpha_n \]

\noindent where $\alpha_i$ are atomic formulae. Note that we have excluded any
universally quantified variables for clarity.

The Herbrand base $\hat{H}$ of $\Pi$ is the set of all facts that can
be expressed using the predicates and constants in $\Pi$; for a program $\Pi =
\langle R, F \rangle$, $F \subseteq \hat{H}$. Note that we do not
distinguish between the predicates in $F$ and those in the heads of
the rules in $R$ (as extensional and intensional predicates), as is
the case in the model theory for Datalog presented in
\cite{datalog-basics}; in this respect, our treatment follows that of
the model theory for Prolog presented in \cite{prolog-semantics}. A
Herbrand interpretation $\mathcal{I}$ of $\Pi$ is any subset of
$\hat{H}$, and a Herbrand interpretation $\mathcal{I}$ that satisfies
the set of clauses in program $\Pi$ is a Herbrand model of $\Pi$,
written $\models_{\mathcal{I}} \Pi$.

We define the concept of entailment in Datalog as follows. A fact $e$
is entailed by a program $\Pi$ iff each interpretation satisfying
$\Pi$ (satisfying every clause in $\Pi$) also satisfies $e$, written
$\Pi \models e$. For a program $\Pi$ and a set of entailed facts
$E$, $\Pi \models E$ iff $\Pi \models e$ for every $e \in E$. For
convenience, we refer to the set of all facts that are entailed by a
program (the intersection of all the Herbrand models of the program)
as the {\em entailment of a program}, written $cons(\Pi)$ and defined
as:
\begin{align*}
  cons(\Pi) &= \{ e \in \hat{H} \mid \Pi \models e \} \\
  &= \bigcap_\mathcal{I} \models_\mathcal{I} \Pi
\end{align*}
\end{definition}

\begin{definition}[Justification]
\labeldef{continuous datalog: justification}

We define a {\em justification} of a set of facts that are entailed by
a Datalog program as a minimal subset of the program that is
sufficient for the entailment to hold. More precisely, for a Datalog
program $\Pi = \langle R, F \rangle$ and a set of entailed facts $E$
where $\langle R, F \rangle \models E$, the program $\langle J_R, J_F
\rangle$ with $J_R \subseteq R$ and $J_F \subseteq F$ is a
justification for $E$ iff $\langle J_R, J_F \rangle \models E$ and,
for all $J'_R \subset J_R$ or $J'_F \subset J_F$, $\langle J'_R, J'_F
\rangle \not\models E$.

We simplify this notion of justification by making the assumption that
a justification can be defined in terms of a minimal subset of the
ground facts of the program; the rules of the program are therefore
implicitly considered to be part of every justification. Any
subsequent references to {\em justification} are to this simplified
sense, which is defined as follows: given a Datalog program $\Pi =
\langle R, F \rangle$ and a set of entailed facts $E$ where $\langle
R, F \rangle \models E$, the set of facts $J \subseteq F$ is a
justification for $E$ iff $\langle R, J \rangle \models E$ and, for
all $J' \subset J$, $\langle R, J' \rangle \not\models E$. Note that a
given set of entailed facts may not have a unique justification.

\end{definition}

\begin{definition}[Time Domain]
\labeldef{continuous datalog: instant}

Let $\mathbb{T}$ be a discrete time domain with a total order under
$\leqslant$; an element of $\mathbb{T}$ is an {\em instant}.
$\mathbb{T}$ has a lower bound $0$, such that for all $i \in
\mathbb{T}$, $0 \leqslant i$. For convenience, we write $i+n$ where $i
\in \mathbb{T}$ to indicate the $n$th successor of instant $i$,
and $i-n$ to indicate the $n$th predecessor of instant $i$.

\end{definition}

\begin{definition}[Fact Instance]
 \labeldef{continuous datalog: fact instance}

A {\em fact instance} is a fact with some additional temporal
annotation regarding the instant from which it is recognised to be true;
it is a tuple containing a fact $f$ and an instant $t \in
\mathbb{T}$, written $\langle f, t \rangle$. Two fact instances
containing the same fact and timestamp are considered identical; two
fact instances containing the same fact but different timestamps are
considered to be distinct.

We represent the relationship between a fact (without temporal
annotation) and a fact instance (with temporal annotation) with the
function $strip()$ which maps a fact instance to its underlying fact:
\[ strip(\langle f, t\rangle) = f \]

For convenience, we also write $strip(F^T) = F$, where $F^T$ is a set
of fact instances and $F = \{ strip(f) \mid f \in F^T \}$.

\end{definition}

In \refdef{continuous datalog: datalog program}, we took a Datalog
program to consist of a set of rules and a set of facts.  In order to
extend this definition of Datalog programs to incorporate fact
instances, we define a Time-Annotated Datalog program as follows:

\begin{definition}[Time-Annotated Datalog]
\labeldef{continuous datalog: ta datalog program}

A {\em Time-Annotated Datalog program} (henceforth abbreviated
TA-Datalog program) $\Pi^T = \langle R, F^T\rangle$ is a tuple which
comprises a finite set of rules $R$ and a finite set of fact instances
$F^T \subseteq \hat{H} \times \mathbb{T}$. Fact instances are as
defined in \refdef{continuous datalog: fact instance}, and rules are
of the same form as in conventional Datalog programs.

When considering a TA-Datalog program, in which there may be many
distinct instances of a given fact, temporal information regarding
when instances of that fact became true would be lost if such a
program were to entail sets of plain facts without temporal
annotations. We therefore extend the model theory for Datalog from
\refdef{continuous datalog: datalog program} by considering the
Herbrand base of all fact instances $\hat{H}^T$ (all facts with all
instants). As before, a fact instance $e$ is entailed by a program
$\Pi^T$ iff each interpretation satisfying every clause in $\Pi^T$
also satisfies $e$, written $\Pi^T \models e$. The {\em entailment of the
TA-Datalog program} $\Pi^T$ is therefore:
\[ cons^T(\Pi^T) = \{ e \in \hat{H}^T \mid \Pi^T \models e \} \]

Informally, the set of instants appearing in some set $E^T$ of fact
instances entailed by some program ${\langle R, F^T \rangle}$ will be
a subset of the instants appearing in the fact instances in $F^T$.  In
addition, the instant $t$ in a fact instance $\langle e, t \rangle$
that is entailed by a TA-Datalog program $\Pi^T = \langle R^T, F^T
\rangle$ will be bounded from above by the maximum instant in any fact
instance in $F^T$. A TA-Datalog program may entail fact instances that
are only distinguished by their instants (and are therefore considered
distinct from each other, as per \refdef{continuous datalog: fact
  instance}).

Consider the following example TA-Datalog program:
\[
\Pi^T = \langle \{ c \Leftarrow a \land b \}, \{ \langle a, 1 \rangle, \langle a, 2 \rangle, \langle b, 1 \rangle, \langle b, 2 \rangle  \} \rangle
\]

The entailment of the program $\Pi^T$ will be:
\[cons^T(\Pi^T) = \{\langle c, 1 \rangle, \langle c, 2 \rangle\}\]

It follows that applying the ${strip()}$ function to all the fact
instances in both the set of fact instances $F^T$ in some ${\Pi^T =
  \langle R, F^T \rangle}$ and the entailment of the program $\Pi^T$
(the set of fact instances $cons^T(\Pi^T)$) maintains the relationship
of entailment; if $\langle R, F^T\rangle \models \langle e, t
\rangle$, then $\langle R, strip(F^T) \rangle \models strip(\langle e,
t \rangle)$. This relationship is shown in \reffig{TD-D}.

\begin{figure}[htb]
  \[
  \begin{CD}
\Pi^T = \langle R, F^T \rangle @>\models>> cons^T(\Pi^T) \\
@V{strip}VV @V{strip}VV \\
\Pi = \langle R, F \rangle @>\models>> cons(\Pi)
  \end{CD}
  \]
\caption{Relationship between Datalog and TA-Datalog entailments}
\labelfig{TD-D}
\end{figure}

Reconsidering the example TA-Datalog program above, we may reduce it
to the following equivalent conventional Datalog program $\Pi$:
\begin{align*}
\Pi &= \langle \{ c \Leftarrow a \land b \}, strip(\{ \langle a, 1 \rangle, \langle a, 2 \rangle, \langle b, 1 \rangle, \langle b, 2 \rangle \}) \rangle \\
    &= \langle \{ c \Leftarrow a \land b \}, \{ a, b \} \rangle
\end{align*}
Trivially, the entailment of this program $\Pi$ will therefore be:
\begin{align*}
cons(\Pi) &= strip(\{\langle c, 1 \rangle, \langle c, 2 \rangle\}) \\
 &= \{ c \}  
\end{align*}
\end{definition}

\begin{definition}[Time-Annotated Justification]
\labeldef{continuous datalog: instance justification}

The justification of entailments is equally applicable to TA-Datalog
programs, with facts in the entailment and the justification for that
entailment being replaced with fact instances.  Following the
simplified notion of justification in \refdef{continuous datalog:
  justification}, we define the justification of a set of fact
instances that are entailed by a TA-Datalog as a minimal subset of
the ground fact instances of the program that are sufficient for
the entailment to hold.

More formally, for a TA-Datalog program $\Pi^T = \langle R, F^T \rangle$
and a set of entailed fact instances $E^T$ where $\langle R, F^T \rangle \models E^T$, the set of fact instances $J^T \subseteq F^T$ is a justification for
$E^T$ iff $\langle R, J^T \rangle \models E^T$ and, for all $J^{T\prime} \subset J^T$, $\langle R, J^{T\prime}\rangle \not\models E^T$.

For an entailed fact instance $\langle e, t\rangle$ and a
justification of that fact instance $J^T$, the instant $t$ at which
$\langle e, t \rangle$ becomes true is taken to be the latest instant
at which any of the fact instances in the justification became true:
\[ t = max(\{ t' \mid \langle j', t' \rangle \in J^T \}) \]

Note that, although an entailed fact instance $\langle e, t\rangle$
may have multiple justifications, all the justifications for $\langle
e, t \rangle$ will share the same maximum instant; a justification
with a different maximum instant $t' \neq t$ would entail a different
fact instance $\langle e, t' \rangle$. Considering again the example
TA-Datalog program in \refdef{continuous datalog: ta datalog program},
the entailed fact instance $\langle c, 2 \rangle$ would have the
following set of justifications:
\[
  \{ \ \{ \langle a, 1 \rangle, \langle b, 2 \rangle \},  \{ \langle a, 2 \rangle, \langle b, 1 \rangle \},  \{ \langle a, 2 \rangle, \langle b, 2 \rangle \} \  \}
\]

\end{definition}

Having defined the notion of time annotation in Datalog, we now define
the notions of streams and windows before extending to Continuous Datalog.

\begin{definition}[Streams]
\labeldef{continuous datalog: stream}

A {\em stream} $S$ is an unbounded sequence of fact instances. We
define a sequence as a mapping from fact instances to instants, where
the instant records the time at which the fact instance appears on the
stream. This has the consequence that a particular fact instance
(i.e. a fact accompanied by a particular instant that is its time
annotation) may appear only once in a given stream. For convenience,
we indicate that a fact instance $\langle f, t\rangle$ appears in $S$
by writing $\langle f, t \rangle \in S$. We indicate that a fact
instance $f_i$ appears strictly before a fact instance $f_j$ on stream
$S$ (i.e. $S(f_i) < S(f_j)$) by writing $f_i \prec f_j$.

Streams may be (but are not necessarily) ordered by the instants in
their fact instances, with the instants non-decreasing in the stream,
such that for any two fact instances $\langle f_i, t_i \rangle$ and
$\langle f_j, t_j\rangle$ where $\langle f_i, t_i \rangle \prec
\langle f_j, t_j \rangle$, $t_i < t_j$; we refer to such
streams as {\em in-order streams}, and to streams whose fact instances
are not ordered by their instants as {\em out-of-order streams}.
\end{definition}

\begin{definition}[Window]
\labeldef{continuous datalog: window}

A window of range ${r \in \mathbb{N}}$ over a stream $S$ at an instant
${i \in \mathbb{T}}$ is the \emph{set} of fact instances that appear in
$S$ in the interval ${(i-r,i]}$, written ${W^{S,r}_{i}}$, and defined as follows:
\[
W^{S,r}_i = \{ \langle f, t \rangle \mid \langle f, t \rangle \in S \land (i - r) < S(\langle f, t\rangle) \leqslant i \}
\]  
\end{definition}

The following axioms define the semantics of streams and windows over
those streams in Continuous Datalog:

{\nobreak\begin{axiom}\label{axiom:continuous datalog: window range leq 0}
A window of range $0$ over a stream $S$ at an instant 
${i \in \mathbb{T}}$ is always empty.
\begin{equation*}
W^{S,0}_{i} = \varnothing
\end{equation*}
\end{axiom}}

{\nobreak\begin{axiom}\label{axiom:continuous datalog: window start leq 0}
A window of any range $x \in \mathbb{N}$ over a stream $S$ at (or before)
the first instant of time is always empty.
\begin{equation*}
W^{S,x}_{i} = \varnothing
\end{equation*}
where $i \leqslant 0$.
\end{axiom}}

{\nobreak\begin{axiom}\label{axiom:continuous datalog: window composition}
A window of range ${x > 0}$ over a stream $S$ at an instant 
${i \in \mathbb{T}}$ is equal to the union of the $x$ consecutive
windows of range $1$ over the stream $S$ preceding and including that at
instant $i$.
\begin{equation*}
W^{S,x}_{i} = \mathop{\cup}_{j=0}^{x-1} W^{S,1}_{i-j}
\end{equation*}
\end{axiom}}

{\nobreak\begin{axiom}\label{axiom:continuous datalog: window disjointness}
A window of range ${x \in \mathbb{N}}$ over a stream $S$ at an
instant ${i \in \mathbb{T}}$ is disjoint from a second window of 
range ${y \in \mathbb{N}}$ over $S$ at any instant before $i - x$.
\begin{equation*}
W^{S,x}_{i} \cap W^{S,y}_{i-z} = \varnothing
\end{equation*}
where $z \geqslant x$.
\end{axiom}}

Given these definitions, a Continuous Datalog program may be formally
defined, both in terms of streams and in terms of its instantaneous
states:

\begin{definition}[Continuous Datalog Program]
\labeldef{continuous datalog: CDP}

We initially take a {\em Continuous Datalog program} (henceforth a
C-Datalog program) $\Pi^C = \langle R, F^0, S \rangle$ to be a tuple
which comprises a finite set of rules $R$, a finite set of static fact
instances $F^0$ (all of which have the instant $0$) and an in-order
stream of fact instances $S$.

The model theory for a such a C-Datalog program broadly follows that
for a TA-Datalog program; we again take the Herbrand base of all fact
instances $\hat{H}$. Disregarding the order of fact instances in $S$,
a fact instance $e$ is entailed by a program $\Pi^C$ iff each
interpretation satisfying every clause in $\Pi^C$ also satisfies $e$.
However, we take the {\em entailment of the C-Datalog program} $\Pi^C$
to be an in-order stream $E^S$, where the order in which entailed fact
instances appear in $E^S$ depends on the order in which the fact
instances in their justifications appear in $S$. Given that that the
instant of an entailed fact instance is the maximum of the instants in
the fact instances that form the justification for that entailed fact
instance (as in \refdef{continuous datalog: instance justification}),
$E^S$ will also be an in-order stream.

However, this definition of a C-Datalog program does not take into
account any notion of windowing over the stream $S$. We define a {\em
  Windowed Continuous Datalog program} $\Pi^{C, r} = \langle R, F^0, S,
r \rangle$ to be a tuple with components as defined above for a
C-Datalog program, but with the addition of $r \in \mathbb{N}$ for the
range of the window over $S$, which effectively specifies that the
fact instances from $S$ are to contribute to the entailments of $\Pi^{C,r}$
from their instant for a duration of $r$.

We consider the entailment of the WC-Datalog program $\Pi^{C,r}$ not
as a single stream of fact instances (as for C-Datalog), but in three
parts: a set of fact instances $E^0 = cons^T(\langle R, F^0\rangle)$
that is the entailment of the TA-Datalog program $\langle R,
F^0\rangle$; a stream $P$ of fact instances entailed by the rules $R$ and
the fact instances in $F^0$ and $S$, ordered by the instant at which they
became justified (i.e. their instants); and a stream $N$ of those same fact
instances, ordered by the instant at which they ceased to be justified:
$\langle R, F^0, S, r \rangle \models \langle E^0, P, N \rangle$. $P$
and $N$ can be considered to be the continuous reasoning equivalents of
the streams produced by the \emph{IStream} and\emph{DStream} operators
respectively of the \emph{CQL} continuous query language \citep{CQL},
which create streams based on the additions to and deletions from a
relation. Note that $P$ is an in-order stream whose members are a subset
of those in $E^S$ for an equivalent C-Datalog program, while $N$ may be
an out-of-order stream.
\end{definition}

From the definition of entailment for TA-Datalog in \refdef{continuous
  datalog: ta datalog program} and the definition of windows in
\refdef{continuous datalog: window}, we can derive the following:

{\nobreak\begin{axiom}\label{axiom:continuous datalog: entailment disjointness}
If two sliding windows over the same in-order stream $S$ cover
non-overlapping time periods, then the set of fact instances entailed
by ${\langle R, W^{S,x}_i \rangle}$ is necessarily disjoint from the
set of fact instances entailed by $\langle R, W^{S,y}_{i-z} \rangle$,
where ${z \geqslant x}$.
\[
\{ e \mid \langle R , W^{S,x}_i \rangle \models e \} \cap \{ e \mid \langle R , W^{S,y}_{i-z} \rangle \models e \} = \varnothing
\]
\end{axiom}}

\begin{definition}[Instanteous State of a WC-Datalog program]
\label{def:continuous datalog: CDPt}

We can reduce a WC-Datalog program $\Pi^{C,r} = \langle R, F^0, S, r
\rangle$ at an instant $i \in \mathbb{T}$ to its TA-Datalog equivalent by
taking the window $W^{S,r}_i$ over stream $S$ of range $r \in \mathbb{N}$;
we refer to this as the {\em instantaneous state} of the WC-Datalog program,
written as $\Pi^{C,r}_i = \langle R, F^0 \cup W^{S,r}_i \rangle$. As
previously shown in \refdef{continuous datalog: ta datalog program}, this
may then be further reduced to the Datalog program $\Pi = \langle R, strip(F^0 \cup W^{S,r}_i) \rangle$.

The instantaneous entailment of a $\Pi^{C,r}_i$ may be divided into two
semantically distinct portions: the \emph{persistent} portion is the
entailment of ${\langle R, F^0 \rangle \models E^0}$, as in a WC-Datalog
program, which is guaranteed to be part of the entailment of every state
due to the monotonicity of positive Datalog; the \emph{transitive} portion
is the set $E^T_{i}$ of entailed fact instances whose justifications are at
least partially contained in $W^{S,r}_i$. {\nobreak We define $E^T_i$ as:
\[
E^T_i = \{ x \mid \Pi^{C,r}_i \models x \land J^T \cup W^{S,r}_i \neq \varnothing \}
\]
where $J^T$ is a justification for the entailment of x by
$\Pi^{C,r}_i$.} By this definition and Axiom~\ref{axiom:continuous
  datalog: entailment disjointness}, $E^0$ is guaranteed to be
disjoint from every $E^T_i$ for every state $\Pi^{C,r}_i$ of every
WC-Datalog program $\Pi^{C,r}$.
\end{definition}

\begin{figure*}
\centering
$
\begin{CD}
  {\Pi^C = \langle R, F, S \rangle} @>\models>> {E^S} \\
  @V{add\ window}VV \\
        {\Pi^{C,r} = \langle R, F^0, S, r \rangle} @>\models>> {\langle cons^T(\langle R, F^0\rangle), P, N \rangle} \\
        @V{evaluate\ window}VV @VV{evaluate\ window}V \\
        {\Pi^{C,r}_i = \langle R, F^0 \cup W^{S,r}_i \rangle} @>\models>> {cons^T(\langle R, F^0 \rangle) \cup (W^{P,r}_i \setminus W^{N,r}_i)} \\
  @VVV @VVV \\
        {\Pi^T = \langle R, F^T \rangle } @>\models>> {cons^T(\Pi^T)} \\
  @V{strip}VV @VV{strip}V \\
        {\Pi = \langle R, F \rangle } @>\models>> {cons(\Pi)}
\end{CD}
$
\caption{Commutative diagram from Continuous Datalog programs to Datalog entailments.}
\labelfig{CD-TD-D}
\end{figure*}

The following axioms define the semantics of the streams of
entailments $P$ and and negations $N$ of a $\Pi^{C,r}$ in terms of the
instantaneous entailments of the sequence of states of that continuous
program.

{\nobreak\begin{axiom}
\label{axiom:continuous datalog: positive window increment}
The window of range 1 at any given instant ${i \in \mathbb{T}}$ over
the stream $P$ entailed by a WC-Datalog program $\Pi^{C,r}$ is defined
as the set of fact instances that are entailed by the state
$\Pi^{C,r}_i$ that were not entailed by the state $\Pi^{C,r}_{i-1}$.
\begin{equation*}
W^{P,1}_{i} = \left( E^0 \cup E^T_{i} \right) \setminus
\left( E^0 \cup E^T_{i-1} \right) = E^T_{i} \setminus E^T_{i-1}
\end{equation*}
\end{axiom}}

{\nobreak\begin{axiom}
\label{axiom:continuous datalog: negative window increment}
The window of range 1 at any given instant ${i \in \mathbb{T}}$ over
the stream $N$ entailed by a WC-Datalog program $\Pi^{C,r}$ is defined
as the set of fact instances that were entailed by the state
$\Pi^{C,r}_{i-1}$ that are no longer entailed by the state
$\Pi^{C,r}_i$.
\begin{equation*}
W^{N,1}_{i} = \left( E^0 \cup E^T_{i-1} \right) \setminus
\left( E^0 \cup E^T_{i} \right) = E^T_{i-1} \setminus E^T_{i}
\end{equation*}
\end{axiom}}

{\nobreak\begin{axiom}
\label{axiom:continuous datalog: entailment precedes negation}
Fact instances may only be negated by some part of their justification
leaving the sliding window $W^{S,r}_{i}$, this necessarily being after
they were entailed.  As such, a window of any range ${x \in
  \mathbb{N}}$ over the stream of entailments $P$ at any given instant
${i \in \mathbb{T}}$ is disjoint from a window of any range ${y \in
  \mathbb{N}}$ over the stream of entailment negations $N$ at any
instant before ${i - x}$.
\begin{equation*}
  W^{P,x}_{i} \cap W^{N,y}_{i-z} = \varnothing
\end{equation*}
where $z \geqslant x$.
\end{axiom}}

As noted above in \refdef{continuous datalog: CDP}, the stream of
negations $N$ of a WC-Datalog program $\Pi^{C,r}$ may be an
out-of-order stream in which fact instances appear on the stream in
the order in which they cease to have any complete justifications.

Thus far the translation from instantaneous sets of entailments to
streams of entailments has been stated axiomatically as the definition
of those streams.

However, it follows from these axioms that the instantaneous set
$E^T_{i}$ of instances entailed at any given instant $i$ by any
$\Pi^{C,r} = \langle R, F^0, S, r \rangle$ may be reconstructed from
windows of sufficient range over its entailed streams $P$ and $N$ at
$i$:
\[ E^T_{i} = W^{P,x}_{i} \setminus W^{N,y}_{i} \]
where ${y \geqslant x \geqslant r}$. The minimum range $x$ is such to achieve
the \emph{complete} set of entailment instances $E^T_{i}$ of the
$\Pi^{C,r}_{i}$. In addition, the minimum range $y$ is such to achieve the
\emph{sound} set $E^T_{i}$. As such, a \emph{valid window} that
contains only and exactly those entailed instances $E^T_{i}$ at every
instant $i$ may be trivially maintained from the streams $P$ and $N$
continuously entailed by some $\Pi^{C,r}$.

In conclusion, given a Continuous Datalog program $\Pi^C = \langle R,
F^0, S \rangle$, we can apply to it a window of range $r$ to yield a
Windowed Continuous Datalog program $\Pi^{C,r} = \langle R, F^0, S, r
\rangle$. We can evaluate this window at an instant $i \in \mathbb{T}$
to yield a Time-Annotated Datalog program $\Pi^T = \langle R, F^0 \cup
W^{S,r}_i \rangle$, from we which can then strip the time annotations
to yield a conventional Datalog program $\Pi = \langle R, F
\rangle$. The relationship between these *-Datalog programs and their
entailments is expressed in the commutative diagram in
\reffig{CD-TD-D}.

\end{nestedsection}
