\begin{nestedsection}{Continuous Datalog}{semantics}
The first requirement of stream reasoning is a well-defined semantics.
This must take into account the continuous nature of a subset of the
base axioms of a stream reasoning problem, as well as its
instantaneous truth and entailments, in line with the work of the
reasoning community.  However, it must also provide a continuous
interpretation of those instantaneous entailments in order to
integrate with other stream-processing technologies.  More
specifically, an appropriate semantics for stream reasoning on the
Semantic Web must: allow the expression of OWL 2 ontologies to some
degree of expressivity; not sacrifice the precise semantics thereof
whilst incorporating the semantics of stream processing; and allow for
the continuous derivation of entailments in a concise manner.

	Continuous Datalog is the extension of positive Datalog with stream-processing semantics, with a view to accommodating the aims of the latter within the well-defined semantics of the former, which are shared with RIF-Core and therefore support reasoning at the expressivity of OWL 2 RL.
	This comprises the extension of Datalog programs to include sources of continuously generated/reported facts, and the application of sliding window semantics over those sources, thereby providing for the interpretation of Continuous Datalog programs as sequences of instantaneous states that are each valid positive Datalog programs.
	As such, these states each inherit the semantics of positive Datalog.
	Furthermore, it involves the continuous derivation of entailments from this sequence of instantaneous states as a set of streams of facts;
	these streams share a data model with the streams of continuous queries, as defined by the stream processing community, thereby allowing the integration of Continuous Datalog programs into stream processing pipelines.

In addition, it can be shown that these streams may be trivially
manipulated to retrieve the instantaneous sets of entailments produced
by the more semantically well-defined offerings from the reasoning
community.

%% fixme
%	In regards to Continuous Datalog, an ``instant'', a ``stream of facts'' and a ``sliding window'' over a stream are defined as follows:

\begin{definition}[Datalog Program]
\label{def:continuous datalog: datalog program}
A conventional Datalog program $\Pi = \langle R, F\rangle$ is a tuple
which comprises a finite set of rules $R$ and a finite set of facts
$F$. Facts are extensional atomic formulae of first-order logic, and
rules are of the form:

\[\alpha_0 :- \alpha_1 , \ldots , \alpha_n\]

\noindent where $\alpha_0$ is an intensional atomic formula and
$\alpha_1 \ldots \alpha_n$ are a mixture of intensional and
extensional atomic formulae.

For the purpose of defining Continuous Datalog in the context of the
Semantic Web, we treat facts as 3-tuples $\langle s, p, o \rangle$,
where $s, p, o$ are respectively the subject, predicate and object of
an RDF triple. 
\end{definition}

\begin{definition}[Instant]
\label{def:continuous datalog: instant}
An instant is a discrete point in time, such that an instant 
${i \in \mathbb{N}}$.

It should be noted that time is a continuous concept, with infinitely
many divisions possible between any two discrete instants.  However,
the resolution with which such divisions can be recognised will, in
practise, have a lower bound set by hardware limitations, or may be
defined as part of some specific implementation.  In these cases,
anything that occurs between discrete instants is considered to have
occurred at the next recognised instant.
\end{definition}

\begin{definition}[Fact Instance]
\label{def:continuous datalog: fact instance}
A {\em fact instance} is a fact with some additional temporal
annotation; it is the combination of a fact's signature (its predicate
$p$, subject $s$ and object $o$) and a timestamp $t$ (an instant) that
is used to record the time at which the fact was asserted or
entailed. A fact instance is denoted $\langle s, p, o \rangle \! : \!
t$, and is unique.

We represent the relationship between a fact (without temporal
annotation) and a fact instance (with temporal annotation) with the
function $strip$ which maps a fact instance to its underlying fact.
\end{definition}

\begin{definition}[Stream of Facts]
\label{def:continuous datalog: stream}
A stream of facts $S$ is a \emph{sequence} of fact instances ordered
by their timestamps; the distinction between multiple instances of a
fact within a stream is represented in 
Axiom~\ref{axiom:continuous datalog: window disjointness}.
\end{definition}

\begin{definition}[Sliding Window]
\label{def:continuous datalog: window}
A sliding window of range ${x \in \mathbb{N}}$ over any stream $S$ at
any given instant ${i \in \mathbb{N}}$ is the \emph{set} of fact
instances that occur in $S$ during the interval ${(i-x,i]}$, and is
  denoted ${W^{S,x}_{i}}$.
\end{definition}

The following axioms define the semantics of streams and windows over
those streams in Continuous Datalog:

\begin{axiom}\label{axiom:continuous datalog: window range leq 0}
A window of range $0$ over a stream $S$ at an instant 
${i \in \mathbb{N}}$ is always empty.
\begin{equation*}
W^{S,0}_{i} = \varnothing
\end{equation*}
\end{axiom}

\begin{axiom}\label{axiom:continuous datalog: window composition}
A window of range ${x \ge 0}$ over a stream $S$ at an instant 
${i \in \mathbb{N}}$ is equal to the union of the $x$ consecutive
windows of range $1$ over the stream $S$ preceding and including that at
instant $i$.
\begin{equation*}
W^{S,x}_{i} = \mathop{\cup}_{j=0}^{x-1} W^{S,1}_{i-j}
\end{equation*}
\end{axiom}

\begin{axiom}\label{axiom:continuous datalog: window disjointness}
A window of range ${x \in \mathbb{N}}$ over a stream $S$ at an
instant ${i \in \mathbb{N}}$ is disjoint from a second window of 
range ${y \in \mathbb{N}}$ over $S$ at any instant before $i - x$.

\begin{equation*}
W^{S,x}_{i} \cap W^{S,y}_{i-z} = \varnothing
\end{equation*}

\noindent where $z \geq x$.
\end{axiom}

In Definition~\ref{def:continuous datalog: datalog program}, we took a
Datalog program to consist of a set of rules and a set of facts.
Taken together, the set of facts $F$ entails another set of factss $E$
in which the existence of one fact in $E$ may be justified by many
distinct subsets of $F$.

When considering streams of facts, in which there may be many distinct
instances of a given fact, temporal information would be lost if sets
of fact instances (denoted $F_{inst}$) were to entail sets of plain
facts.  As such, where a fact $f$ would be entailed by some set of
facts $F$ that are instantiated on some stream, an instance of $f$
should be entailed by the set of instances of those facts present on
that stream.  Furthermore, each such justification in $F_{inst}$ for a
fact entails a unique instance of that fact, such that each entailed
instance has a single justification within the original set of facts,
and many instances of the entailed fact may occur in the entailed set
of facts $E_{inst}$.  This form of entailment shall be henceforth
referred to as {\em instance entailment}, and shall be the norm when
discussing the entailments of sets of fact instances.
	
It should be noted that removing all temporal information from a set
of fact instances $F_{inst}$ produces the set of facts $F$
instantiated in $F_{inst}$:

\[F = \{ strip(x) | x \in F_{inst} \}\]

In addition, applying the ${strip()}$ function to both the original
fact instances $F_{inst}$ and the entailed fact instances $E_{inst}$
maintains the relationship of entailment between the two sets:

\[ \left( F_{inst} \vDash E_{inst} \right) \rightarrow \left( F \vDash E \right) \]

From this description of instance entailment,
Axiom~\ref{axiom:continuous datalog: entailment disjointness} may be
derived:

\begin{axiom}
\label{axiom:continuous datalog: entailment disjointness}

If two sets of fact instances, $A_{inst}$ and $B_{inst}$, are
disjoint, then a set $F_{inst1}$ of entailed fact instances (each of
which is justified at least in part by $A_{inst}$) is necessarily
disjoint from any set $F_{inst2}$ of entailed fact instances (each
justified at least in part by $B_{inst}$), because entailed fact
instances are distinct with regards to the fact instances that justify
them.

\begin{multline*}
\left( A_{inst} \cap B_{inst} = \varnothing \right) \wedge \left( A_{inst} \vDash E_{inst} \right) \\
\wedge \left( B_{inst} \vDash F_{inst} \right) \rightarrow \left( E_{inst} \cap F_{inst} = \varnothing \right)
\end{multline*}
\end{axiom}

Given these definitions, a Continuous Datalog program may be formally
defined, both in terms of streams and in terms of its instantaneous
states:

\begin{definition}[Continuous Datalog Program]
\label{def:continuous datalog: CDP}

A Continuous Datalog Program (CDP) is a tuple $\langle R, F, S, r
\rangle$ which comprises a finite set of rules $R$, a finite set of
facts $F$, a stream of facts $S$ and a range $r \in \mathbb{N}$ for
the window over $S$. This specifies that the streamed data from $S$ is
to be considered valid from its arrival for the duration represented
by $r$.

The entailment of such a program is composed of a stream of entailed
instances $P$ and a stream of instance negations $N$, whose contents
are defined in 
Axioms~\ref{axiom:continuous datalog: positive window increment}, 
\ref{axiom:continuous datalog: negative window increment} and
\ref{axiom:continuous datalog: entailment precedes negation}. 
The streams $P$ and $N$ are therefore the continuous
reasoning equivalents of the streams produced by the \emph{IStream}
and \emph{DStream} operators of the \emph{CQL} continuous query
language \citep{CQL} respectively.  The truth value of a CDP is
instantaneous, with the the truth function of Continuous Datalog
$T^{CD}$ taking both a program ${CDP}$ and an instant ${i \in
  \mathbb{N}}$ as arguments.
\end{definition}

\begin{definition}[State of a CDP]
\label{def:continuous datalog: CDPt}

The state of a CDP at a given instant $i$ (denoted as ${CDP_{i}}$) is
the union of the set of facts $F$ and the window $W^{S,r}_{i}$.  As both $F$
and ${strip( W^{S,r}_{i} )}$ are sets of Datalog axioms, 
${CDP_{i}}$ constitutes a valid Datalog program that may be evaluated
by the truth function of Datalog $T^{D}$, the result of which is the
definition for the instantaneous truth of the ${CDP}$ at instant $i$
by $T^{CD}$.  The instantaneous entailments of a $CDP_{i}$ are the
union of the set $E$, the persistent entailments of $F$ (which are
guaranteed to be entailed by every state, as Positive Datalog is
monotonic), and the set $E_{i}$ of instances of the transient
entailments that are justified at least in part by some subset of
$W^{S,r}_{i}$.

\begin{equation*}
T^{CD} \left( CDP, i \right) = T^{D} \left( CDP_{i} \right)
\end{equation*}
\begin{equation*}
CDP_{i} = F \cup W^{S,r}_{i} \vDash E \cup E_{i}
\end{equation*}
\end{definition}


The following axioms define the semantics of the streams of
entailments $P$ and and negations $N$ of a ${CDP}$ in terms of the
instantaneous entailments of the sequence of states of that continuous
program.

\begin{axiom}
\label{axiom:continuous datalog: positive window increment}

The window of range 1 at any given instant ${i \in \mathbb{N}}$ over
the stream $P$ entailed by a ${CDP}$ is defined as the set of facts
that are entailed by the state of the ${CDP}$ at instant $i$ that were
not entailed by the state of ${CDP}$ at the previous instant instance
${i-1}$.

\begin{equation*}
W^{P,1}_{i} = \left( E \cup E_{i} \right) \setminus \left( E \cup E_{i-1} \right) = E_{i} \setminus E_{i-1}
\end{equation*}
\end{axiom}

\begin{axiom}
\label{axiom:continuous datalog: negative window increment}
The window of range 1 at any given instant ${i \in \mathbb{N}}$ over
the stream $N$ entailed by a ${CDP}$ is defined as the set of facts
that were entailed by the state of the ${CDP}$ at the previous instant
instance ${i-1}$ that are no longer entailed by the state of ${CDP}$
at instant $i$.
\begin{equation*}
W^{N,1}_{i} = \left( A' \cup E_{i-1} \right) \setminus \left( A' \cup E_{i} \right) = E_{i-1} \setminus E_{i}
\end{equation*}
\end{axiom}

\begin{axiom}
\label{axiom:continuous datalog: entailment precedes negation}
Facts may only be negated by some part of their justification leaving
the sliding window $W^{S,r}_{i}$, this necessarily being after they
were entailed.  As such, a window of any range ${x \in \mathbb{N}}$
over the stream of entailments $P$ at any given instant ${i \in
  \mathbb{N}}$ is disjoint from a window of any range ${x \in
  \mathbb{N}}$ over the the stream of entailment negations $N$ at any
instant after ${i - X}$

CHECK

\begin{equation*}
\forall X' \geq X \left( W^{P,x}_{i} \cap W^{N,y}_{i-z} = \varnothing \right)
\end{equation*}

\noindent where $z \geq x$.
\end{axiom}

From this, we can see that each instance of an entailment in some
$E_{i}$ may take the same form as an instance of any other fact in a
stream of facts, being $\langle s, p, o \rangle \! : \!t$, where $t$ is
the real time associated with the instant $i_{e}$ in which
$\langle s, p, o \rangle$ becomes newly justified by the ${CDP}$ (more
specifically, by some subset of the instances in $W^{S,r}_{i_{e}}$).
As such, the stream of entailment instances $P$ of a ${CDP}$ takes
the same form as the \emph{streams of facts} already described, thus
producing a concise stream of entailments that may be read by
further stream processing.  In addition, the stream of negations $N$
of a ${CDP}$ also takes this form, but differs from the streams
discussed thus far in that it is not ordered by the entailment time
$i_{e}$ of the entailment instances but by the time at which their
justification \emph{ceases} to be valid (i.e. the first instant
${i_{n} \geq i_{e}}$ in which $W^{S,r}_{i_{n}}$ no longer contains
the subset of fact instances first present in $W^{S,r}_{i_{e}}$ that
justified the entailment instance $\langle s, p, o \rangle \! : \!t$).

Thus far the translation from instantaneous sets of entailments to
streams of entailments has been stated axiomatically as the definition
of those streams.  However, it can be shown from these axioms, as in
\ref{sec:semantics: proof for entailments}, that it is also possible
to reconstruct exactly the instantaneous set $E_{i}$ of entailment
instances of any ${CDP}$ at any given instant from the windows of
sufficient range over its entailed streams $P$ and $N$ at that
instant.

CHECK

\[ \forall Y \geq X \geq R \, \left( E_{i} = W^{P,X}_{i} \setminus W^{N,Y}_{i} \right) \]

The minimum range $x$ is such to achieve the \emph{complete} set of
entailment instances $E_{i}$ of the ${CDP}_{i}$.  In addition, the
minimum range $y$ is such to achieve the \emph{correct} set $E_{i}$.
As such, a \emph{valid window} that contains only and exactly those
instances of entailments in $E_{i}$ at every instant $i$ may be
trivially maintained from the streams $P$ and $N$ continuously
entailed by some ${CDP}$.
\end{nestedsection}
